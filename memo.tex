\documentclass[UTF8]{ctexart} % for Chinese input
\usepackage{amsmath}
\begin{document}
\section{}
传统做法:把bias加入原状态变量,形成增广状态矩阵,叫做Augmented state Kalman filter (ASKF)。计算困难:矩阵维数增长导致的计算复杂度和截断误差。

\cite{friedlandTreatmentBiasRecursive1969}最早提出把ASKF解耦,即把偏差和状态隔离开的思想,即Two stage Kalman filter(TSKF),也叫bias separate Kalman filter.考虑模型

\[\begin{array}{l}
{{\bf{x}}_{k + 1}} = F{{\bf{x}}_k} + B{{\bf{b}}_k} + {v_x}\\
{b_{k + 1}} = {b_k}\\
{z_k} = H{x_k} + D{b_k} + w
\end{array}\]	  	

该方法的本质思想就是把ASKF解耦成两个滤波器,分别称为bias-free filter和bias filter。

如果考虑bias的噪声,ASKF的解耦遇到了困难。原来的方法不是最优的,即和原来的ASKF不等价。在解耦的过程中做了一些近似,称为VDB和KVDB技术。 %要验证

在多目标多传感器跟踪的背景下,\cite{xiangdonglinExactMultisensorDynamic004}利用多传感器的测量得到精确的bias filter,并利用多目标的测量来提高bias的估计精度,称为EX(exact)方法。\cite{taghaviMultisensorMultitargetBearing2016}针对被动测量的情况进行了讨论。坐标转换技巧在被动测量下意义不大,转换后依然是非线性的。这种方法,主动测量至少需要两个传感器,被动测量至少需要3个传感器。在目标个数较少的时候,效果会降低。

% TODO: use this method
\cite{chien-shuhsiehOptimalSolutionTwostage1999}修改了bias free filter,增加了bias的补偿,实现了最优。在此基础上,\cite{chenNovelTwostageExtended2016}针对非线性系统,提出了EKF版本。\cite{wufanJiYuATSUKFSuanFaDeWeiXingZiKongXiTongGuZhangGuJi2018}提出了UKF版本。

% TODO: use this method
后来那本书提出了Gaussian mean shift registration方法来估计bias。

\equiv
\end{document}